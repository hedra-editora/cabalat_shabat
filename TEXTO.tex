\chapter*{Nota da organizadora}
\addcontentsline{toc}{chapter}{Nota da organizadora, \emph{por Fabiana Gampel Grinberg}}


\begin{flushright}
\emph{Fabiana Gampel Grinberg}
\end{flushright}

Essa obra é fruto da vontade de conectar o desejo de manter a tradição
judaica com a necessidade de ressignificá-la. Aqui estão selecionadas
bênçãos e canções tradicionais para receber o shabat, traduzidas com
sinceridade e reflexão.

Através do texto em português, este pequeno guia se propõe a acolher um
ponto de vista amplo da espiritualidade, vasta e abundante demais para
se restringir a um modo de expressão rígido --- e não se atém ao pé da
letra dos registros religiosos. \emph{Abençoado seja o Espírito do
Universo, Fonte da Eternidade, que nos dá a habilidade de questionar}.

Este trabalho só foi possível graças à amizade da Suzana Salama, do
Gabriel Neistein e da Marília Neustein.

Todá rabá!

\chapter*{Introdução ao \emph{cabalat shabat}}
\addcontentsline{toc}{chapter}{Introdução ao \emph{cabalat shabat}, \emph{por Gabriel Neistein}}


\begin{flushright}
\emph{Gabriel Neistein}
\end{flushright}

Do hebraico, טאלאבק (cabalat) vem de הלאבק (cabalá), ou seja, receber. A
origem da cerimônia de cabalat shabat nasce na mística de Isaac Luria,
cabalista de Tzfat (ou Safed) do século XVI, e simboliza o recebimento
da schechiná. Apresentada sob a figura de uma noiva em Lechá Dodi, é uma
presença serena que conduz a paz nas sinagogas e espaços privados.

Em caráter introdutório à noção do shabat, reúno duas passagens. A
primeira, que segue abaixo, é uma pequena história chassídica recolhida
por Martin Buber em suas viagens pelos shtetls:

\begin{quote}
\emph{Semana após semana, com a chegada do shabat, os irmãos Rabi Zússia
e o Rabi Elimelech eram tomados de grande sentimento de santidade. Uma
vez disse o Rabi Elimelech ao Rabi Zússia: }

\emph{--- Irmão, às vezes tenho medo de que meu sentimento de santidade
no shabat não seja verdadeiro, que seja apenas imaginação. }

\emph{--- Irmão -- disse Zússia --- eu também tenho, às vezes, este
medo.}

\emph{--- O que vamos fazer? - perguntou Elimelech. }

\emph{Zússia respondeu: }

\emph{-- Vamos cada um de nós, num dia qualquer da semana, preparar uma
refeição, exatamente igual ao jantar de shabat, sentar-nos entre os
chassidim e dizer palavras dos ensinamentos. }

\emph{Assim fizeram: prepararam uma completa refeição de shabat,
vestiram roupas limpas, puseram os gorros de pele, comeram no meio dos
chassidim e disseram palavras dos ensinamentos. Então desceu sobre eles
um imenso sentimento de santidade, como se fosse shabat.}
\end{quote}

Podemos interpretar, a partir do pequeno conto, que a espiritualidade
está ao alcance humano através da prática, dos encontros e dos
compartilhamentos significativos da temporalidade. Desse modo nos
envolvemos em santidade.

Em seguida, transcrevo uma passagem do livro \emph{O shabat}, de Abraham
J. Heschel:

\begin{quote}
\emph{Este exato momento pertence a todos os homens vivos, tal como me
pertence. Nós partilhamos o tempo, nós possuímos o espaço. Pelo fato de
eu possuir o espaço, sou um rival de todos os outros seres; através da
minha existência no tempo, eu sou um contemporâneo de todos os outros
vivos.}

\emph{O significado do shabat é, antes, o de celebrar o tempo, e não o
espaço. Seis dias da semana vivemos sob a tirania das coisas do espaço;
no Shabat tentamos nos tornar harmônicos com a santidade no tempo. É um
dia em que somos chamados a partilhar no que é eterno no tempo, para
fugir dos resultados da criação, para os mistérios da criação; do mundo
da criação para a criação do mundo.}
\end{quote}

O recebimento do shabat e sua vivência durante a noite da sexta-feira e
dia do sábado podem ser entendidos como uma catedral indestrutível da
arquitetura do tempo para os judeus, conforme diz Heschel no mesmo livro
citado acima. Guardar o shabat é também guardar o tempo, que implica em
manter-se alheio às ações dos outros seis dias da semana, permitindo à
alma o descanso necessário.

\chapter*{}
\addcontentsline{toc}{part}{Cabalat Shabat}
\begin{center}
\begin{vplace}[0.3]
\Large
Cabalat Shabat
\end{vplace}
\end{center}
\thispagestyle{empty}


\movetoevenpage
\raggedleft

% Alinhar os tiutlos dos poemas com o proprio poema
\section{Bênção das velas}

בָּרוּך אַתָּה אַדָנָי

אֱלהֵינוּ מֶלֶך הָעוֹלָם

אַשֶׁר קִדְשָנוּ בְּמִצְוֹתָיו

וְצִוָנוּ לְהַדְלִיק נֵר שֶל שַבָּת‏\footnote{Baruch atá Adonai,/ Eloheinu melech haolam,/ asher kidshanu bemitzvotáv,/ vetzivanu lehadlic ner shel shabat.}

\movetooddpage
\raggedright

\section{}

Bendito sejas tu,

Eterno nosso D'us, \emph{Rei do Universo},

que nos santificaste com teus mandamentos,

e nos ordenaste acender as velas de shabat.

\movetoevenpage
\raggedleft

\section{Iedid Nefesh}

יְדִיד נֶפֶשׁ, אָב הָרַחְמָן

מְשךְ עַבְדָךְ אֶל רְצונָךְ

יָרוּץ עַבְדָךְ כְמו אַיָל

יִשְתַחֲוֶה מוּל הֲדָרָךְ

כִּי יֶעְרַב לוֹ יְדִידוּתָךְ

מִנּפֶת צוּף וְכָל טָעַם\\[10pt]

הָדוּר, נָאֶה, זִיו הָעולָם

נַפְשִי חולַת אַהֲבָתָךְ

אָנָא אֵל נָא, רְפָא נָא לָהּ

בְּהַרְאות לָהּ נעַם זִיוָךְ

אָז תִתְחֵזֵּק וְתִתְרַפֵּא

וְהָיְתָה לָךְ שִׂמְחַת עולָם\\[10pt]

וָתִיק יֶהֱמוּ נָא רַחֲמֶיךָ

וְחוּס נָא עַל בֵּן אוֹהֲבָךְ

כִּי זֶה כַּמֶּה נִכְסוֹף נִכְסַפְתִּי

לִרְאות בְּתִפְאֶרֶת עֻזָךְ

אָנָא אֵלִי, מַחְמָד לִבִי

חוּסה נָא, וְאַל תִּתְעַלָם


\movetooddpage
\raggedright

\section{}

\emph{Amado da minha alma},

me chame em sua direção,

Correrei como um animal,

pois quero apreciar toda a sua majestade.

E receber seu afeto é para mim,

mais doce que todo o mel.\\[10pt]

\emph{Fonte de toda a glória que há no mundo},

Minha alma arde de amor por \emph{Você},

Por favor, cure-a,

mostre-me toda a beleza de seu esplendor.

Então serei fortalecida,

e será minha toda a alegria.\\[10pt]

\emph{Fonte da Eternidade},

Me acolha como uma criança,

Pois tanta é a minha vontade,

de admirar todo seu poder.

Este é o desejo do meu coração,

não se esconda.


\movetoevenpage
\raggedleft


הִגָלֵה נָא וּפְרשׂ, חָבִיב

עָלַי אֶת סֻכַת שְלומֶךְ

תָּאִיר אֶרֶץ מִכְּבוֹדָךְ

נָגִילָה וְנִשְׂמְחָה בָךְ

מַהֵר אָהוּב, כִּי בָא מועֵד

וְחָנֵנִי כִּימֵי עולָם\footnote{
    % Colocar as transcricoes por estrofe. Uma nota por estrofe
Iedid nefesh av harachaman,/ meshoch avdecha el retzonecha,/ iarutz avdecha kemo aial,/
ishtachavê el mul hadarecha./ Ki ierav lo iedidotecha,/ minofet tzuf vechol taam.//
Hadur naê ziv haolam,/ nafshi cholat ahavatecha,/ Ana El na, refá na la,/ beharot la noam zivechá./ Az titchazek vetitrapê,/ vehaietah la simchat olam.// Vatik iehemu na rachamecha,/
vechusá na al ben ahuvecha,/ ki ze cama nichsof nichsafti,/ lirot betiferet uzechá,/
ana eli machmad libi,/ vechusá na veal titalam.// Higaleh na ufrós chavivi alai,/
et sucat shelomecha/ Tair eretz mikvodecha,/ nagila venismechah bach./ Maher ahuv ki va moed,/
vechoneinu kimei olam.}

\movetooddpage
\raggedright


Revele-se, \emph{Fonte Eterna do Amor},

e estenda sobre mim um manto de paz,

Que a \emph{Eterna Fonte de Luz} ilumine todo o mundo,

que todos possam brilhar de alegria.

Depressa, \emph{meu amado}, esta é a hora,

se aproxime e me abrace pela eternidade.

\movetoevenpage
\raggedleft


\section{Lechá Dodi}

לְכָה דוֹדִי לִקְרַאת כַּלָּה

פְּנֵי שַׁבָּת נְקַבְּלָה\\[10pt]

שָׁמוֹר וְזָכוֹר בְּדִבּוּר אֶחָד

הִשְׁמִיעָנוּ אֵל הַמְּיֻחָד

ה' אֶחָד וּשְׁמוֹ אֶחָד

לְשֵׁם וּלְתִפְאֶרֶת וְלִתְהִלָּה\\[10pt]

לִקְרַאת שַׁבָּת לְכוּ וְנֵלְכָה

כִּי הִיא מְקוֹר הַבְּרָכָה

מֵרֹאשׁ מִקֶּדֶם נְסוּכָה

סוֹף מַעֲשֶּׂה בְּמַחֲשָׁבָה תְּחִלָּה\\[10pt]

מִקְדַּשׁ מֶלֶךְ עִיר מְלוּכָה

קוּמִי צְאִי מִתּוֹךְ הַהֲפֵכָה

רַב לָךְ שֶׁבֶת בְּעֵמֶק הַבָּכָא

וְהוּא יַחֲמוֹל עָלַיִךְ חֶמְלָּה\\[10pt]

הִתְעוֹרְרִי הִתְעוֹרְרִי

כִּי בָא אורֵךְ קוּמִי אוֹרִי

עוּרִי עוּרִי שִׁיר דַּבֵּרִי

כְּבוֹד ה' עָלַיִךְ נִגְלָּה


\movetooddpage
\raggedright

\section{}

Vem, \emph{meu amado},

encontrar a noiva.

Venha receber a presença,

do shabat.\\[10pt]

\emph{Guardar e Lembrar} são

duas palavras em uma expressão,

O \emph{Eterno} nos revela que é \emph{Um},

que \emph{Seu nome} é \emph{Único} e por isso cantamos.\\[10pt]

Vamos receber o shabat,

que é a origem de todas as bênçãos,

Desde o início do tempo,

o último feito é o primeiro pensamento.\\[10pt]

Desperta, acorda,

pois resplandece o brilho,

Levanta e entoa a melodia,

os raios da luz nos aquecem.

\movetoevenpage
\raggedleft

בֹּאִי בְשָׁלוֹם עֲטֶרֶת בַּעְלָהּ

גַּם בְּשִּׂמְחָה וּבְצָהֳלָה

תּוֹךְ אֱמוּנֵי עַם סְגֻלָּה

בּוֹאִי כַלָּה בּוֹאִי כַלָּה\footnote{Lechá dodi licrat calá,/ penei shabat necabelá.// Shamor vezachor bedibur echad,/ Hishmianu el hameiuchad./ Adonai echad ushemó echad,/ Leshem uletiferet veletehila.//
Licrat shabat lechu venelchá/ Ki hi mecor haberachá,/ merosh mikedem nesuchá,/ sof maassê bemachshavá tehilá.// Hitoreri, hitoreri,/ Ki va orech cumi ori,/ Uri uri shir daberi,/
Kvod Adonai alaich niglá.// Boi veshalom ateret baalá,/ Gam besimchá uvetsarlá,/
Toch emunei am segulá,/ Boi calá, boi calá./}

\movetooddpage
\raggedright

Venha em paz, noiva,

com música e alegria,

Te recebemos com apreço,

venha noiva, venha noiva.

\movetoevenpage
\raggedleft

\section{Shalom Aleichem}

שָלוֹם עֲלֵיכֶם מַלְאֲכֵי הַשָרֵת מַלְאֲכֵי עֶלְיוֹן

מִמֶלֶךְ מַלְכֵי הַמְלָכִים הַקָדוֹשׁ בָרוּךְ הוּא\\[10pt]

בּוֹאֲכֶם לְשָׁלוֹם מַלְאֲכֵי הַשָּׁלוֹם מַלְאֲכֵי עֶלְיוֹן

מִמֶלֶךְ מַלְכֵי הַמְלָכִים הַקָדוֹשׁ בָרוּךְ הוּא\\[10pt]

בָרְכוּנִי לְשָלוֹם מַלְאֲכֵי הַשָּׁלוֹם מַלְאָכִי עֶלְיוֹן

מִמֶלֶךְ מַלְכֵי הַמְלָכִים הַקָדוֹשׁ בָרוּךְ הוּא\\[10pt]

צֵאתְכֶם לְשָלוֹם מַלְאֲכֵי הַשָּׁלוֹם מַלְאָכִי עֶלְיוֹן

מִמֶלֶךְ מַלְכֵי הַמְלָכִים הַקָדוֹשׁ בָרוּךְ הוּא\footnote{
Shalom alechem malachei, hasharet malachei Elion,/
mimelech malchei hamelachim hacadosh Baruch Hu.//
Boachem leshalom malachei, hashalom malachei Elion,/
mimelech malchei hamelachim hacadosh Baruch Hu//
Barechuni leshalom malachei, hashalom malachei Elion,/
mimelech malchei hamelachim hacadosh Baruch Hu.//
Tsetechem leshalom malachei, hasharet malachei Elion,/
mimelech malchei hamelachim hacadosh Baruch Hu.}

\movetooddpage
\raggedright

\section{}

Estejam em paz, anjos protetores,\qb{} mensageiros do infinito,

da suprema \emph{Divindade},  \qb{}do que é soberano, do que é abençoado.\\[10pt]

Que venham em paz, os anjos da paz,\qb{} mensageiros do infinito,

da suprema \emph{Divindade},\qb{} do que é soberano, do que é abençoado.\\[10pt]

Abençoem-me com a paz,\qb{} anjos da paz, mensageiros do infinito,

da suprema \emph{Divindade},\qb{} do que é soberano, do que é xabençoado.\\[10pt]

Que partam em paz, os anjos\qb{} protetores, mensageiros do infinito,

da suprema \emph{Divindade}, \qb{}do que é soberano, do que é abençoado.

\movetoevenpage
\raggedleft

\section{Kidush}

וַיְכֻלּוּ הַשָּׁמַיִם וְהָאָרֶץ וְכָל צְבָאָם

וַיְכַל אֱלהִים בַּיּום הַשְּׁבִיעִי מְלַאכְתּו אֲשֶׁר עָשָׂה

וַיִּשְׁבּת בַּיּום הַשְּׁבִיעִי מִכָּל מְלַאכְתּו אֲשֶׁר עָשָׂה

וַיְבָרֶךְ אֱלהִים אֶת יום הַשְּׁבִיעִי וַיְקַדֵּשׁ אתו

כִּי בו שָׁבַת מִכָּל מְלַאכְתּו אֲשֶׁר בָּרָא אֱלהִים לַעֲשׂות\\[10pt]

בָּרוּךְ אַתָּה אַדָנָי אֱלהֵינוּ מֶלֶךְ הָעולָם בּורֵא פְּרִי הַגָּפֶן\\[10pt]

בָּרוּךְ אַתָּה ה' אֱלהֵינוּ מֶלֶךְ הָעולָם

אֲשֶׁר קִדְּשָׁנוּ בְּמִצְותָיו וְרָצָה בָנוּ. וְשַׁבַּת קָדְשׁו
בְּאַהֲבָה וּבְרָצון הִנְחִילָנוּ

זִכָּרון לְמַעֲשֵׂה בְרֵאשִׁית

כִּי הוּא יום תְּחִלָּה לְמִקְרָאֵי קדֶשׁ זֵכֶר לִיצִיאַת מִצְרָיִם\\[10pt]

כִּי בָנוּ בָחַרְתָּ וְאותָנוּ קִדַּשְׁתָּ מִכָּל הָעַמִּים וְשַׁבַּת
קָדְשְׁךָ בְּאַהֲבָה וּבְרָצון הִנְחַלְתָּנוּ

בָּרוּךְ אַתָּה אַדָנָי מְקַדֵּשׁ הַשַּׁבָּת\footnote{
Vaichulu hashamaim vehaaretz vechol tzevaam,/
Vaichal Elohim baiom hashvií melachtó asher asá,/
Vaishbot baiom hashvií micol melachtó asher asá,/
Vaivarech Elohim et iom hashvií vaicadeish oto,/
Ki vo shavat micol melachtó asher bará Elohim laasot.//
Baruch atá, Adonai Eloheinu, Melech haolam, borê peri hagafen.//
Baruch atá, Adonai Eloheinu, Melech haolam, asher kideshanu bemitzvotav veratza/
vanu, ve shabat codsho beahavá uveratzon hinchilanu, zikaron lemaasê vereishit./
Ki hu iom techila lemikraei codesh, zecher litziat Mitzrayim.//
Ki vanu vacharta, veotanu kidashta, micol haamim,/
Ve shabat codshechá beahava uveratzon hinchaltanu,/
Baruch atá, Adonai, mecadêsh ha shabat./}

\movetooddpage
\raggedright

\section{}

Os céus e a terra e todos que vivem lá foram criados,

Completou-se no sétimo dia a obra da \emph{Fonte da Criação},

E no sétimo dia contemplou-se todo o trabalho de \qb{}então,

E este dia foi escolhido e abençoado,

pois foi quando houve descanso

de tudo o que havia para ser criado.\\[10pt]

Abençoada seja a \emph{Eterna Fonte da Vida} que faz crescer \qb{}o fruto da videira.\\[10pt]

Abençoada seja a \emph{Eterna Fonte de Força}, que nos \qb{}permite escolher nossas ações,

Recebemos o shabat com amor, como herança e como memória da \emph{Eterna
Fonte da Criação}. Escolhemos o shabat como a lembrança do refúgio de
nossas lutas, como na luta pela liberdade na saída do Egito.\\[10pt]

Abençoada seja a \emph{Eterna Fonte da Vida} que cria o amor, a alegria,
a música e o prazer e nos presenteia com o shabat.

\movetoevenpage
\raggedleft

\section{Benção da chalá}

בָּרוּך אַתָּה אַדָנָי

אֱלהֵינוּ מֶלֶך הָעוֹלָם

הָמוֹציא לֶחם מן הַארץ\footnote{Baruch atá Adonai,/ Eloheinu melech haolam,/ hamotzi lechem min haaretz.}

\movetooddpage
\raggedright

\section{}

Bendito sejas Tu, \emph{Eterno},

nosso D'us, \emph{Rei do Universo,}

que fazes brotar o pão da terra.
