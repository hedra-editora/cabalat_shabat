
%\chapter*{Introdução ao \emph{cabalat shabat}}
%\addcontentsline{toc}{chapter}{Introdução ao \emph{cabalat shabat}, \emph{por %Gabriel Neistein}}
%
%
%\begin{flushright}
%\emph{Gabriel Neistein}
%\end{flushright}
%
%%טאלאבק הלאבק
%%\begin{multicols}{2}
%
%
%O cabalismo surgiu originalmente no sul da França, mas alcançou pleno
%desenvolvimento na Espanha do século \versal{XIII}.  Ganhou entretanto
%novo significado a partir do século \versal{XVI} ao fornecer uma
%espécie de resposta à questão do exílio dos judeus ibéricos em 1492. No
%século \versal{XVII} foi vinculado ao movimento messiânico de Sabatai
%Tzvi, considerado herético à época, e que mesmo em colapso provocou a
%omissão da cabala dos círculos judaicos oficiais.  Isaac Luria, o
%``Ari'', foi o maior e mais conhecido nome dentre os cabalistas,
%originando a partir de seus ensinamentos a Escola Luriânica. Embora
%nascido em Jerusalém, estabeleceu"-se na cidade de Safed --- que se
%tornou o centro cabalista mundial --- por volta de 1569. 
%
%A palavra hebraica קבאלת\,, 
%\emph{cabalat}, vem de \emph{cabalá}, ``recebimento''. A
%origem da cerimônia de \emph{cabalat shabat} nasce na mística de Isaac Luria %(1534--1572),
%cabalista de Tzfat (ou Safed), e simboliza o recebimento
%da \emph{Schechiná}, literalmente, ``assentamento'', ``habitação'' ou %``moradia'', que 
%em sentido alegórico significa ``presença de Deus''. 
%
%Apresentada sob a figura de uma noiva no poema \emph{Lechá Dodi},\footnote{Literalmente ``Vem, meu amado''. Alguns ritos judaicos foram influenciados pela cabala, principalmente os que adotaram as bodas sagradas em suas cerimônias --- representadas pela união entre os aspectos femininos e masculinos ou Deus, o rei e a rainha, ou mais precisamente a \emph{Shechiná} e a mística \emph{Ecclesia} de Israel. Popularmente foram entendidas como uma união entre Deus e Israel, o que, para os cabalistas, representa apenas o aspecto externo da interioridade secreta %de Deus. [N.E.]} cantiga do 
%serviço religioso do \textit{cabalat shabat}, trata-se de uma
%presença serena que conduz à paz nas sinagogas e espaços privados.
%
%Em caráter introdutório à noção de \emph{shabat}, reúno a seguir duas
%passagens sobre o termo. A primeira diz respeito a uma pequena história %chassídica 
%recolhida pelo filósofo e escritor  
%Martin Buber (1878--1975), em suas viagens pelos \emph{shtetls}:\footnote{%Povoações ou bairros 
%judaicos da Europa Oriental. [N.E.]}
%
%\begin{quote}
%Semana após semana, com a chegada do \emph{shabat}, os irmãos Rabi Zússia
%e o Rabi Elimelech eram tomados de grande sentimento de santidade. Uma
%vez disse o Rabi Elimelech ao Rabi Zússia:
%
%--- Irmão, às vezes tenho medo de que meu sentimento de santidade
%no \emph{shabat} não seja verdadeiro, que seja apenas imaginação.
%
%--- Irmão -- disse Zússia --- eu também tenho, às vezes, este medo.
%
%--- O que vamos fazer? --- perguntou Elimelech.
%
%Zússia respondeu:
%
%--- Vamos cada um de nós, num dia qualquer da semana, preparar uma
%refeição, exatamente igual ao jantar de \emph{shabat}, sentar"-nos entre os
%\emph{chassidim} [tementes a Deus] e dizer palavras dos ensinamentos.
%
%Assim fizeram: prepararam uma completa refeição de \emph{shabat},
%vestiram roupas limpas, puseram os gorros de pele, comeram no meio dos
%\emph{chassidim} e disseram palavras dos ensinamentos. Então desceu sobre eles
%um imenso sentimento de santidade, como se fosse \emph{shabat}.\footnote{\emph{Histórias do Rabi}, Martin Buber (tradução M. Arnsdorff, T. Belinky, J. %Guinsburg, R. Mautner, R. Schivartche, R. Simis), Editora Perspectiva, 2000.}
%\end{quote}
%
%O pequeno texto pode ser entendido como uma busca pela essência
%do \emph{shabat} e a diferença entre esse dia e o resto da semana. No judaísmo é o anoitecer, amanhecer ou entardecer que trazem a santidade. E a espiritualidade do sétimo dia vai de encontro às formas significantes do tempo: ``o \emph{shabat} não é uma data, mas uma atmosfera''.\footnote{\emph{O schabat: seu significado para o homem moderno}, Abraham Joshua Heschel (tradução Fany Kon \& Jacob %Guinsburg), Editora Perspectiva, 2000, p.~32.}
%
%Em seguida, transcrevo uma passagem do livro \emph{O schabat}, do rabino e %teólogo 
%Abraham J. Heschel (1907--1972):
%
%\begin{quote}
%Este exato momento pertence a todos os homens vivos, tal como me
%pertence. Nós partilhamos o tempo, nós possuímos o espaço. Pelo fato de
%eu possuir o espaço, sou um rival de todos os outros seres; através da
%minha existência no tempo, eu sou um contemporâneo de todos os outros
%vivos.
%
%O significado do \emph{shabat} é, antes, o de celebrar o tempo, e não o
%espaço. Seis dias da semana vivemos sob a tirania das coisas do espaço;
%no \emph{shabat} tentamos nos tornar harmônicos com a santidade no tempo. É um
%dia em que somos chamados a partilhar no que é eterno no tempo, para
%fugir dos resultados da criação, para os mistérios da criação; do mundo
%da criação para a criação do mundo.\footnote{Idem, p.~18.}
%\end{quote}
%
%Falamos aqui de um tempo festivo, meditativo e simplesmente prazeroso  --- conceito apresentado no mesmo livro de Heschel, exposto acima e em notas anteriores. Guardar o \emph{shabat} é, portanto, guardar o tempo. Mas o mais importante é lembrar que ``estamos dentro do \emph{shabat} mais do que o \emph{%shabat} está dentro de nós''.\footnote{Idem, p.~32.}
%
%%\end{multicols}
\thispagestyle{empty}
\movetooddpage

%\vspace*{1cm}

\begin{verse}
\textsc{eu canto em hinos}\\[15pt]

Eu canto em hinos\\
para entrar pelos portões\\
do campo das maçãs,\\
das sagradas maçãs.\\[5pt]

Uma mesa nova\\
pomos para ela,\\
um belo candelabro\\
derrama sua luz sobre nós.\\[5pt]

Entre a direita e a esquerda\\
a Noiva se aproxima\\
em joias sagradas\\
e vestes festivas.\\[5pt]

Seu esposo abraça-a\\
em seus fundamentos,\\
dá-lhe satisfação,\\
espreme sua força.\\[5pt]

Tormentos e gritos\\
são passados.\\
Há rostos novos agora\\
e almas e espíritos.\\[5pt]

Ele lhe dá alegria\\
em redobrada medida.\\
Luzes brilham\\
e rios de bênção.\\[5pt]

Padrinhos, avançai\\
e preparei a noiva,\\
provisões de muitas espécies\\
e todos os tipos de peixe.\\[5pt]

A fim de gerar novas almas\\
e espíritos novos\\
nos trinta e dois caminhos\\
e três ramos.\\[5pt]

Ela tem setenta coroas\\
mas acima dela o Rei,\\
para que todos sejam coroados\\
no Santos dos Santos.\\[5pt]

Todos os mundos são formados\\
e selados dentro dela,\\
mas todos brilham\\
do Ancião dos Dias.\\[5pt]

Em direção ao sul coloco\\
o candelabro místico,\\
Faço lugar ao norte\\
para a mesa com os pães.\\[5pt]

Com vinho nos cálices\\
e fardos de mirto\\
para fortificar os Noivos\\
pois eles estão fracos.\\[5pt]

Nós lhes trançamos coroas\\
de palavras preciosas\\
para a coroação dos setenta\\
em cinquenta portais.\\[5pt]

Deixai a Schechiná ser envolvida\\
pois seis filões sabáticos\\
ligados por todos os lados\\
com o Santuário Celestial.\\[5pt]

Enfraquecidas e banidas\\
as forças impuras,\\
os demônios ameaçadores\\
agora estão em grilhões.
\end{verse}

\begin{flushright}
Isaac Lúria\footnote{Isaac Lúria, o ``Ari'' (1534--1572), foi o maior e mais conhecido nome dentre os cabalistas, originando a partir de seus ensinamentos a Escola Luriânica. Nascido em Jerusalém, estabeleceu-se na cidade de Tzfat (ou Safed) --- que se tornou o centro cabalista mundial --- por volta de 1569. A palavra hebraica \emph{cabalat}, vem de \emph{cabalá}, ``recebimento''. A cerimônia de \emph{cabalat shabat} como a conhecemos hoje tem grande parte na mística de Lúria e simboliza o recebimento da \emph{Schechiná}, literalmente, ``assentamento'', ``habitação'' ou ``moradia'', que em sentido alegórico significa ``presença de Deus''. A cabala surgiu originalmente no sul da França, mas alcançou pleno desenvolvimento na Espanha do século \versal{XIII}. Ganhou novo significado a partir do século \versal{XVI} ao ser resignificada como uma espécie de resposta à questão do exílio dos judeus ibéricos em 1492. No século \versal{XVII} foi vinculada ao movimento messiânico de Sabatai Tzvi, considerado herético à época, e que mesmo em colapso provocou a omissão da cabala dos círculos judaicos oficiais.}
\end{flushright}

\chapter*{A celebração do casamento}
\addcontentsline{toc}{chapter}{A celebração do casamento, \emph{por Jorge Sallum}}

\begin{flushright}
\emph{Jorge Sallum}
\end{flushright}

O conjunto de poemas que ora apresentamos são chamados em hebraico de
\emph{piyyut} ou \emph{piyut} (no plural \emph{piyyutim} ou
\emph{piyutim}), termo derivado do grego \emph{poiētḗs}, que significa
simplesmente ``poeta''. Na
teoria poética grega do período imperial, o gênero dos poemas incluídos
pela tradição nessa recolha é denominado poesia epitalâmia ou
epitalâmica. O termo significa ``diante do tálamo'' ou seja, do ``quarto
de dormir''.

%\footnote{\emph{Shulchan Arukh}, ``Orach Chayim'' 584.}

A poesia epitalâmia se apresenta em um lugar, no qual mulheres
preparam e declamam poemas momentos antes de deixar os noivos a sós para
um primeiro encontro amoroso. Na raríssima definição sobre esse gênero
poético que nos chegou da antiguidade, lemos claramente que ``os versos
epitalâmios eram cantados aos recém"-casados por meninas e virgens,
diante do quarto''.\footnote{Albert Severyns. \emph{Recherches sur la Chrestomathie de Proclos: le codex 239 de Photius}. Paris: Les Belles Lettres, 1953, §62.}

Simbolicamente, o \emph{shabat} encena algo semelhante: a celebração do
casamento entre Deus e o povo de Israel. A tradição cabalista do \emph{shabat}
deu contornos bastante específicos ao ritual e transformou"-o em
alegoria, utilizando-se para isso desse lugar poético. Segundo Gershom
Scholem\footnote{Gerhard Scholem tornou"-se Gershom Scholem após a imigração para Israel. Filósofo e historiador alemão, foi um dos maiores estudiosos das correntes místicas do judaísmo. Relacionou"-se de maneira próxima com Leo Strauss, Walter Benjamin e Theodor Adorno, sendo que a correspondência trocada com os dois últimos foi publicada. [\versal{N.~E.}]} revela com precisão sobre a popularidade da alusão às bodas
sagradas dentre os cabalistas, ``Israel Najara, o poeta do círculo de
Safed, redigiu um contrato de matrimônio poético, provavelmente o
primeiro do gênero --- uma lírica paráfrase mística da certidão de
nascimento pela lei judaica''.\footnote{Gershom Scholem. \emph{A Cabala e seu simbolismo}. São Paulo: Editora Perspectiva, 1978, p. 167-8.} O \emph{shabat} foi assim
ratificado como um ``contrato lírico'' de casamento.

É muito coerente com a performance necessária desse tipo de poema lírico
o fato de o \emph{shabat} ser de fato praticado em casa e não na sinagoga; por
mulheres e com a bênção delas;\footnote{A tradição dita que as mulheres judias acendam duas velas no anoitecer da sexta"-feira e introduzam o \emph{shabat} através da bênção \emph{Hadlakat nerot} ou \emph{Acender das velas}. [\versal{N.~E.}]} como acompanhamento de uma comunhão a ser celebrada entre dois entes que acabaram de se unir, em dia muito
especial. E isto condiz com o que chamamos de \emph{topos} em grego,
\emph{lugar poético}.

Os lugares poéticos da poesia antiga se organizam em cânones. São quatro
as grandes divisões canônicas: épica, jâmbico, elegia, lírica, sendo
\emph{grosso modo} a épica para a guerra; o jâmbico para a disputa; a
elegia para o lamento da morte ou amor não correspondido; e finalmente a
lírica, para celebrar a vitalidade na sua multiplicidade.

Segundo uma das poucas fontes claras da antiguidade sobre gêneros
poéticos --- a \emph{Crestomatia arcaica}, citada acima ---, o
epitalâmio faz parte da lírica e compõe um conjunto de tipos ou gênero
de poemas ao lado do \emph{epinício}, que servia para cantar as
vitórias; do \emph{escólio}, para narrar bebedeiras; da \emph{poesia
erótica}, que não era para a exposição do sexo, mas sim para tratar da
beleza de mulheres, meninos e virgens; do \emph{silo}, para brigas,
invectivas e escárnios; do \emph{treno}, para discursar sobre um morto
recente; do \emph{poema fúnebre}, próximo ao treno, mais específico para
ser executado diante do morto; e, finalmente, do \emph{himeneu}, que
servia basicamente para cantar o casamento no dia da cerimônia,
enquanto se espera o noivo.

Os poemas cantados na cerimônia do \emph{shabat}, caso iniciada na sinagoga,
estão mais próximos do gênero himeneu do que de um epitalâmio,\footnote{Os poemas nessa edição foram dispostos através dessa lógica. Os dois primeiros, \emph{Amado de minha alma} e \emph{Vem, meu amado}, são cantados em sinagogas. \emph{Estejam em paz}, em seguida, indica o caminho protegido por dois anjos que cada judeu toma no início do \emph{shabat} até sua casa, segundo o Talmud. Por fim vêm as bênçãos de introdução doméstica à noite da sexta"-feira: o acendimento das duas velas, a bênção da \emph{chalá} (pão judaico trançado) e o \emph{kidush} (bênção feita com vinho). [\versal{N.~E}.]} pois
trata"-se de um lugar em que a noiva aguarda o noivo, e a força poética
será toda voltada para o desejo de que ele se apresente imediatamente
diante de um público maior. O \emph{Cântico dos cânticos}, associado ao
casamento sagrado, era originalmente entoado nas sinagogas nas tardes de
sexta"-feira e podia ser tomado como um himeneu para a \emph{Schechiná},\footnote{O surgimento da referência às bodas sagradas foi combinado às figuras da Noiva ou da \emph{Schechiná}, presença feminina de Deus, descrita no \emph{Zohar} como a bela virgem que perdeu os olhos de tanto chorar no exílio. Muitas fontes relatam que as preces eram entoadas de olhos fechados por esse motivo. [\versal{N.~E}.]}
somente após o anoitecer eram proferidas as tradicionais preces do
sábado.

Segundo o poeta latino Horácio, a lírica diz respeito ao ``privilégio de
celebrar os deuses, os filhos dos deuses, o púgil vencedor, o cavalo
ganhador da corrida, as inquietações da mocidade e as liberdades
do vinho''. É preciso lembrar que todos os gêneros da lírica que
mencionamos --- epinício, escólio, erótica, silo, treno, fúnebre e
himeneu --- todos esses, estão ligados somente ``às inquietações da
mocidade''. A lírica tem ainda uma segunda linha de gêneros somente para
deuses e uma terceira, para deuses e homens, das quais não trataremos
aqui, como não trataremos tampouco dos outros três grandes cânones, a
saber, a épica, o jâmbico e a elegia.

É interessante notar que os gêneros das ``inquietações da mocidade'' nada
têm a ver com deuses ou homens e deuses a princípio. Deuses gregos
guerreiam, brigam entre si, se imiscuem em assuntos humanos, inspiram
heróis, tornam"-se coléricos, e cada um desses sentimentos têm lugares
poéticos muito específicos. Como é de se esperar de um contexto
religioso politeísta, o jogo de forças e brigas tem função e dinâmica.

O apanhado de poemas presentes nesta edição apresenta na encenação do
casamento pelo \emph{shabat} um Deus mais próximo de quem o procura no âmbito
doméstico. Um deus"-cônjuge jovial, um povo"-consorte a se descobrir, ou
ainda mais em circuito individual, um deus"-torá"-cônjuge a se relacionar
com sua alma"-consorte que deve iniciar"-se no conhecimento. E esse lugar
é reforçado poeticamente pelas características e estratégias específicas
do gênero poético epitalâmio, numa manobra alegórica que transforma um
dos menos divinos gêneros em um lugar de comunhão.



\chapter*{}
\addcontentsline{toc}{part}{Cabalat Shabat}
\begin{center}
\begin{vplace}[0.3]
\Large
\versal{CABALAT SHABAT}
\end{vplace}
\end{center}
\thispagestyle{empty}

\begingroup
\movetoevenpage
\raggedleft

\vspace*{1cm}

\addcontentsline{toc}{chapter}{Iedid Nefesh}
\textsc{iedid nefesh}\\[15pt]

הָרַחְמָן אַב נֶפֶשׁ יְדִיד

רְצוֹנֶךָ אֶל עַבְדְּךָ מְשׁוֹךְ

אַיָּל כְּמוֹ עַבְדְּךָ יָרוּץ

הֲדָרֶךָ מוּל יִשְׁתַּחֲוֶה

יְדִידוֹתֶךָ לוֹ יֶעֱרַב  כִּי

\footnote{Iedid nefesh, av harachamán,/ meshoch avdechá el retzonechá./ iarutz avdechá kemô aial,/ ishtachavê el mul hadarechá./ Ki ierav lo iedidotechá/ minofet tzuf vechol taam.}טָעַם וְכָל צוּף מִנוֹפֶת\\[10pt]

הָעוֹלָם זִיו נָאֶה הָדוּר

אַהֲבָתֶךָ חוֹלַת נַפְשִׁי

לָהּ נָא רְפָא נָא אֵל אָנָּא

זִיוֶךָ נֹעַם לָהּ בְּהַרְאוֹת

וְתִתְרַפֵּא תִּתְחַזֵּק אָז

\footnote{Hadur naê ziv haolam,/ nafshi cholat ahavatechá./ Ana El na, refá na lá,/ beharót la noam zivechá./ Az titchazêk vetitrapê,/ vehaitá la simchat olam.}עוֹלָם שִׂמְחַת לָהּ וְהָיְתָה\\[10pt]

רַחֲמֶיךָ נָא יֶהֱמוּ וָתִיק

אֲהוּבֶךָ בֵּן עַל נָּא וְחוּסָה

נִכְסַפְתִּי נִכְסוֹף כַּמֶּה זֶה כִּי

עֻזֶּךָ בְּתִפְאֶרֶת לִרְאוֹת

לִבִּי חָמְדָה אֵלֶּה אָנָּא

\footnote{Vatik iehemu na rachamechá,/ vechusá na al ben ahuvechá./ Ki ze camá nichsof nichsaftí/ lirot betiféret uzechá./ Ana ele chamdá libi,/ chusá na veal titalêm.}תִּתְעַלָּם וְאַל נָּא חוּסָה\\[10pt]

עָלַי חֲבִיבִי וּפְרוֹשׂ נָא הִגָּלֶה

שְׁלוֹמֶךָ סֻכַּת אֶת

מִכְּבוֹדֶךָ אֶרֶץ תָּאִיר

בָךְ וְנִשְׂמְחָה נָגִילָה

מוֹעֵד בָא כִּי אֱהוֹב מַהֵר

\footnote{Higale na ufrós, chavivi alai,/ et sucat shelomechá./ Tair eretz mikvodechá,/ nagila venismechá bach./ Maher ahuv ki bá moed,/ vechonêinu kimei olam.}עוֹלָם כִּימֵי וְחָנֵּנוּ



\movetooddpage
\raggedright

\vspace*{1cm}

\textsc{amado da minha alma}\\[15pt]

Amado da minha alma, meu querido,

me leve ao seu encontro.

Correrei em sua direção como um cervo,

para admirar toda a sua majestade.

Receber seu afeto é para mim

mais doce que todo o mel.\\[10pt]

Fonte de toda a glória que há no mundo,

minha alma arde de amor por \emph{Você}.

Por favor, cuide dela,

e mostre-me toda a beleza de seu esplendor.

Então serei forte e estarei curado,

e será completa a minha alegria.\\[10pt]

Eterno, seja piedoso,

e poupe seu filho amado.

Pois grande é a minha vontade

de admirar toda a sua glória.

Te suplico, meu Deus, de meu coração desejoso,

não se afaste.\\[10pt]

Revele-se e ilumine, meu querido,

e estenda sobre mim um manto de paz.

E faça brilhar toda a terra com sua honra,

para o nosso prazer e alegria.

Depressa, meu Amado, esta é a hora,

se aproxime e me abrace pela eternidade.

%\movetoevenpage
%\raggedleft %Continuação Iedid Nefesh
%
%\vspace*{1cm}


%\movetooddpage
%\raggedright %continuação amado da minha alma
%
%\vspace*{1cm}



\movetoevenpage
\raggedleft
\addcontentsline{toc}{chapter}{Lechá Dodi}

\vspace*{1cm}

\textsc{lechá dodi}\\[15pt]

כַּלָּה לִקְרַאת דוֹדִי לְכָה

\footnote{Lechá dodi licrat calá,/ pnei shabat necabelá.}נְקַבְּלָה שַׁבָּת פְּנֵי\\[10pt]

אֶחָד בְּדִבּוּר וְזָכוֹר שָׁמוֹר

הַמְּיֻחָד אֵל הִשְׁמִיעָנוּ

אֶחָד וּשְׁמוֹ אֶחָד ה' 

\footnote{Shamor vezachor, bedibur echad,/ hishmianu el hameiuchad./ Adonai echad ushemó echad,/ leshem uletiferet veletehilá.}וְלִתְהִלָּה וּלְתִפְאֶרֶת לְשֵׁם\\[10pt]

וְנֵלְכָה לְכוּ שַׁבָּת לִקְרַאת

הַבְּרָכָה מְקוֹר הִיא כִּי

נְסוּכָה מִקֶּדֶם מֵרֹאשׁ 

\footnote{Licrat shabat lechu venelchá,/ ki hi mecor habrachá./ Merosh mikedem nesuchá/ sof maassê bemachshavá tehilá.}תְּחִלָּה בְּמַחֲשָׁבָה מַעֲשֶּׂה סוֹף\\[10pt]

מְלוּכָה עִיר מֶלֶךְ  מִקְדַּשׁ

הַהֲפֵכָה מִתּוֹךְ צְאִי קוּמִי 

הַבָּכָא בְּעֵמֶק שֶׁבֶת  לָךְ רַב

\footnote{Micdash melech, ir meluchá,/ kumi tzí mitoch haafechá./ Rav lach shevet beemek habachá,/ vehu iachamól alaich chemlá.}חֶמְלָה עָלַיִךְ יַחֲמוֹל  וְהוּא\\[10pt]

קוּמִי מֵעָפָר הִתְנַעֲרִי

עַמִּי תִפְאַרְתֵּךְ בִּגְדֵי לִבְשִׁי 

הַלַּחְמִי בֵּית יִשַׁי בֶּן יַד עַל

\footnote{Hitnaari meafar kumi,/ livshi bigdei tifartech, ami./ Al iad ben Ishai, beit haLachmi,/ karvá el nafshi guealá.}גְאלָּהּ נַפְשִׁי אֶל קָרְבָה\\[10pt]

הִתְעוֹרְרִי הִתְעוֹרְרִי

אוֹרִי קוּמִי אורֵךְ בָא כִּי

דַּבֵּרִי שִׁיר עוּרִי עוּרִי

\footnote{Hitoreri, hitoreri,/ ki ba orech, kumi ori./ Uri uri shir daberi,/ kvod Adonai alaich niglá.}נִגְלָּה עָלַיִךְ ה' כְּבוֹד\\[10pt]


\movetooddpage
\raggedright

\vspace*{1cm}
\textsc{vem, meu amado}\\[15pt]


Vem, meu amado, encontrar a noiva,

venha receber a presença do shabat.\\[10pt]

Guarda e Lembra, em uma só fala,

escuta a voz única.

Ele é Um e o seu nome é Um,

que seu nome seja louvado e celebrado.\\[10pt]

Receba o shabat,

que é a origem de todas as bênçãos.

Desde os tempos mais antigos

a última ação é também o primeiro pensamento.\\[10pt]

Templo santo, cidade real,

levanta e sai das ruínas.

Já não habita mais o Vale de Lágrimas,

e Ele terá compaixão de ti.\\[10pt]

Levanta e desprenda as cinzas,

vistam-se com as roupas de festa, minha nação.

Pela mão do filho de Ishai, de Belém,

minha alma se aproxima da redenção.\\[10pt]


Acorda, acorda,

eis que veio a sua luz, ``levanta minha luz''.

Desperta e entoa a melodia,

dignifique a Deus e Ele se revelará.\\[10pt]


\movetoevenpage
\raggedleft %continuação lechá dodi

\vspace*{1cm}

תִכָּלְמִי וְלא תֵבושי לא 

תֶּהֱמִי וּמַה תִּשתּוחֲחִי מַה

עַמִּי עֲנִיֵּי יֶחֱסוּ בָּךְ

\footnote{Lo tevoshi velo tikalmi,/ má tishtochachi umá tehemi?/ Bach iechesu aniei ami,/ venivnetá ir al tilá.}תִּלָּהּ עַל עִיר וְנִבְנְתָה\\[10pt]

שאסָיִךְ לִמְשסָּה וְהָיוּ

מְבַלְּעָיִךְ כָּל וְרָחֲקוּ

אֱלהָיִךְ עָלַיִךְ יָשיש

\footnote{Vehaiú limshisá shosaich,/ verachaku kol mibalaich./ Iasis alaich Elohaich,/ kimsos chatan al calá.}כַּלָּה עַל חָתָן כִּמְשוש\\[10pt]

תִּפְרוצִי וּשמאל יָמִין

תַּעֲרִיצִי ה' וְאֶת

פַּרְצִי בֶּן אִיש יַד עַל 

\footnote{Iamin usmol tifrotzi,/ vet Adonai taaritzi,/ Al iad ish ben Partzi,/ venismechá venagilá.}וְנָגִילָה וְנִשמְחָה\\[10pt]


בַּעְלָהּ עֲטֶרֶת בְשָׁלוֹם בֹּאִי

וּבְצָהֳלָה בְּשִּׂמְחָה גַּם 

סְגֻלָּה עַם אֱמוּנֵי תּוֹךְ 

\footnote{Boi beshalom, ateret baalá,/ gam besimchá ubetzhalá./ Toch emunei am segulá,/ boi calá, boi calá.}כַלָּה בּוֹאִי כַלָּה בּוֹאִי

\movetooddpage
\raggedright %continuação vem, meu amado

\vspace*{1cm}



Não sinta vergonha ou medo,

por que se abate e lamenta?

Meu povo aflito será poupado,

e a cidade reconstruída sobre a colina.\\[10pt]

Atacados serão os que te fizerem mal,

e seus inimigos, banidos.

Por você seu Deus se encantará,

como um noivo que se alegra com sua noiva.\\[10pt]

Pela direita e pela esquerda

nosso Senhor será celebrado.

Pela mão de do filho de Partzi,

iremos comemorar e cantar.\\[10pt]

Venha em paz, noiva,

com júbilo e alegria.

Te recebemos com apreço,

venha noiva, venha noiva.


\movetoevenpage
\raggedleft
\addcontentsline{toc}{chapter}{Shalom Aleichem}

\vspace*{1cm}

\textsc{shalom aleichem}\\[15pt]

עֶלְיוֹן מַלְאֲכֵי הַשָרֵת מַלְאֲכֵי עֲלֵיכֶם שָלוֹם

\footnote{Shalom aleichem, malachei hasharet, malachei Elion,/ Mimelech malchei hamelachim, hacadosh Baruch Hu.}הוּא בָרוּךְ הַקָדוֹשׁ הַמְלָכִים מַלְכֵי מִמֶלֶךְ\\[10pt]

עֶלְיוֹן מַלְאֲכֵי הַשָּׁלוֹם מַלְאֲכֵי לְשָׁלוֹם בּוֹאֲכֶם

\footnote{Boachem leshalom, malachei hashalom, malachei Elion,/ Mimelech malchei hamelachim, hacadosh Baruch Hu.}הוּא בָרוּךְ הַקָדוֹשׁ הַמְלָכִים מַלְכֵי מִמֶלֶךְ\\[10pt]

עֶלְיוֹן מַלְאָכִי הַשָּׁלוֹם מַלְאֲכֵי לְשָלוֹם בָרְכוּנִי

\footnote{Barechuni leshalom, malachei hashalom, malachei Elion,/ Mimelech malchei hamelachim, hacadosh Baruch Hu.}הוּא בָרוּךְ הַקָדוֹשׁ הַמְלָכִים מַלְכֵי מִמֶלֶךְ\\[10pt] 

עֶלְיוֹן מַלְאָכִי הַשָּׁלוֹם מַלְאֲכֵי לְשָלוֹם צֵאתְכֶם 

\footnote{Tsetechem leshalom, malachei hasharet, malachei Elion,/ Mimelech malchei hamelachim, hacadosh Baruch Hu.}הוּא בָרוּךְ הַקָדוֹשׁ הַמְלָכִים מַלְכֵי מִמֶלֶךְ


\movetooddpage
\raggedright

\vspace*{1cm}

\textsc{estejam em paz}\\[15pt]


Estejam em paz, anjos enviados, anjos do Altíssimo,

do Rei dos reis, santificado seja Ele.\\[10pt]

Que venham em paz, anjos da paz, anjos do Altíssimo,

do Rei dos reis, santificado seja Ele.\\[10pt]

Abençoem-me com a paz, anjos da paz, anjos do Altíssimo,

do Rei dos reis, santificado seja Ele.\\[10pt]

Que partam em paz, os anjos da paz, anjos do Altíssimo,

do Rei dos reis, santificado seja Ele.

\movetoevenpage
\raggedleft



\addcontentsline{toc}{chapter}{Hadlakat Nerot}

\vspace*{1cm}

\textsc{hadlakat nerot}\\[15pt]

אַדָנָי אַתָּה בָּרוּך

הָעוֹלָם מֶלֶך אֱלהֵינוּ

בְּמִצְוֹתָיו קִדְשָנוּ אַשֶׁר

\footnote{Baruch atá Adonai,/ Eloheinu, Melech haolam,/ asher kidshanu bemitzvotáv,/ vetzivánu lehadlic ner shel shabat.}שַבָּת שֶל נֵר לְהַדְלִיק וְצִוָנוּ‏

\movetooddpage
\raggedright

%\addcontentsline{toc}{chapter}{Bênção das velas}
\vspace*{1cm}

\textsc{acender das velas}\\[15pt]

Abençoado seja meu Senhor,

nosso Deus,\footnote{
		%Literalmente ``Abençoado seja o senhor, rei do mundo''. A expressão
		%``Eterna Fonte de Luz'' é uma tradução alegórica, que indica uma imagem menos
		%centralizadora ou hierarquizada de D'us. 
		Muitos termos similares a \textit{Rei do universo} aparecem
		ao longo das bênçãos e poemas rituais. Convém lembrar que o nome de Deus
		no judaísmo não é pronunciado e por isso é possível
		representá"-lo de diversas maneiras, embora seja anunciado por
		epítetos como \textit{Rei do mundo, Rei dos reis, meu Senhor}.
		É preciso entender também que a palavra `Deus' não é o nome de
		Deus, assim como `homem' não é o nome de `Moisés'. [\versal{N.~E.}]} 
Rei do universo,\footnote{\textit{Haolam}, o `mundo' ou `universo', como tudo
		que existe, incluindo o desconhecido; por oposição à \textit{haaretz},
		a `terra', ou `o mundo conhecido', `mundo dos homens', e até `ocupação
		mundana' (ver \textit{Pirkei Avot} 2:2). [\versal{N.~E.}]}

que nos abençoou com seus mandamentos

e nos desejou acender a chama do shabat.\footnote{``A chama do \textit{shabat}'' remete à simbologia das duas velas
	que são acesas no anoitecer da sexta"-feira, e que correspondem às
	expressões \emph{shamor}, ``guarde'' (``Guardar o dia do
	\emph{shabat} para santificá"-lo''. Deuteronômio \versal{V}:~12),
	e \emph{zachor}, ``lembre''
	(``Recordar o dia de \emph{shabat} para santificá"-lo''. Êxodo
	\versal{XX}:~8), mencionadas nos Dez Mandamentos. [\versal{N.~E.}]}

\movetoevenpage
\raggedleft

\addcontentsline{toc}{chapter}{Birkat hamotzi}

\vspace*{1cm}

\textsc{birkat hamotzi}\\[15pt]

אַדָנָי אַתָּה בָּרוּך

הָעוֹלָם מֶלֶך אֱלהֵינוּ 

\footnote{Baruch atá Adonai,/ Eloheinu, Melech haolam,/ hamotzi lechem min haaretz.}הַארץ מן לֶחם הָמוֹציא

\movetooddpage
\raggedright

\vspace*{1cm}

\textsc{bênção do pão que brota da terra}\\[15pt]

Abençoado seja meu Senhor,

nosso Deus, Rei do universo,

que faz brotar o pão da terra.


\movetoevenpage
\raggedleft

\addcontentsline{toc}{chapter}{Kidush
\medskip}

\vspace*{1cm}

\textsc{kidush}\\[15pt]

צְבָאָם וְכָל וְהָאָרֶץ הַשָּׁמַיִם וַיְכֻלּוּ 

עָשָׂה אֲשֶׁר מְלַאכְתּו הַשְּׁבִיעִי בַּיּום אֱלהִים וַיְכַל 

עָשָׂה אֲשֶׁר מְלַאכְתּו מִכָּל הַשְּׁבִיעִי בַּיּום וַיִּשְׁבּת 

אתו וַיְקַדֵּשׁ הַשְּׁבִיעִי יום אֶת אֱלהִים וַיְבָרֶךְ 

\footnote{Vaichulu hashamaim vehaaretz vechol tzevaam,/ vaichal Elohim baiom hashvií melachtó asher asá./ Vaishbot baiom hashvií micol melachtó asher asá,/ vaivarech Elohim et iom hashvií vaicadeish otô,/ ki bô shabat micol melachtó asher bará Elohim laasot.}לַעֲשׂות אֱלהִים בָּרָא אֲשֶׁר מְלַאכְתּו מִכָּל שָׁבַת בו כִּי\\[10pt] 

\footnote{Baruch atá Adonai, Eloheinu, Melech haolam, borê pri hagafen.}הַגָּפֶן פְּרִי בּורֵא הָעולָם מֶלֶךְ אֱלהֵינוּ אַדָנָי אַתָּה בָּרוּךְ\\[10pt] 

הָעולָם מֶלֶךְ אֱלהֵינוּ ה' אַתָּה בָּרוּךְ

בָנוּ וְרָצָה בְּמִצְותָיו קִדְּשָׁנוּ אֲשֶׁר

הִנְחִילָנוּ וּבְרָצון בְּאַהֲבָה קָדְשׁו וְשַׁבַּת

בְרֵאשִׁית לְמַעֲשֵׂה זִכָּרון

\footnote{Baruch atá Adonai, Eloheinu, Melech haolam,/ asher kideshanu bemitzvotáv veratzá banu,/ Ve shabat cadsho beahavá uveratzon hinchilanu,/ zikarón lemaasê bereshit,/ ki hu iom tehilá lemikraei codesh, zecher litziat Mitzraim.}מִצְרָיִם לִיצִיאַת זֵכֶר קדֶשׁ לְמִקְרָאֵי תְּחִלָּה יום הוּא כִּי\\[10pt]

הָעַמִּים מִכָּל קִדַּשְׁתָּ וְאותָנוּ בָחַרְתָּ בָנוּ כִּי

\footnote{Ki banu bachartá veotanu kidashtá micol haamim,/ veshabat kadshechá beahavá uveratzon hinchaltánu.}הִנְחַלְתָּנוּ וּבְרָצון בְּאַהֲבָה קָדְשְׁךָ וְשַׁבַּת \\[10pt]


\footnote{Baruch Atá Adonai, mekadêsh hashabat.}הַשַּׁבָּת מְקַדֵּשׁ אַדָנָי אַתָּה בָּרוּךְ

\movetooddpage
\raggedright

\vspace*{1cm}

\textsc{bênção [do vinho]}\\[15pt]

Foram criados os céus e a terra e todos que lá vivem,

a obra de Deus foi no sétimo dia concluída.

E no sétimo dia contemplou-se toda a criação,

e Deus abençoou o sétimo dia e o santificou,

porque nele houve descanso de tudo o que havia sido criado.\\[10pt]

Abençoado seja, meu Senhor, nosso Deus, que faz crescer o fruto da
videira.\\[10pt]

Abençoado seja meu Senhor, nosso Deus, Rei do mundo,

e que seja santificado o shabat com amor e fervor.

Para santificarmos seus mandamentos Ele nos escolheu,

e lembrar do que foi feito no princípio,

porque este é o dia em que recordamos a saída do Egito.\\[10pt]

E a nós, de todas as nações, o Senhor abençoou,

e santificaremos o shabat com amor e fervor, porque Você nos escolheu.\\[10pt]


Abençoado seja, meu Senhor, santificador do shabat.


%\movetoevenpage
%\raggedleft %continuação kidush

%\vspace*{1cm}



%\movetooddpage
%\raggedright %continuação bênção do vinho

%\vspace*{1cm}




\endgroup

%\chapter*{Nota da organizadora\footnote{ 
	%A publicação surgiu a partir da realização de uma cerimônia de 
%	\emph{cabalat shabat}, proposta pelo Instituto Brasil"-Israel (\versal{IBI}), na qual participantes judeus e não"-judeus puderam encontrar uma forma de expressão, contemplação e exercício de significação própria. Como produtora do \versal{IBI}, me incumbi da tarefa de reunir nesta coletânea bênçãos tradicionalmente recitadas no recebimento do \emph{shabat}, que materializam a preservação da tradição judaica ao %passo que a revisitamos.
%	Este trabalho só foi possível graças à amizade da Suzana Salama, do Gabriel Neistein e %da Marília Neustein.} }
%\addcontentsline{toc}{chapter}{Nota da organizadora, \emph{por Fabiana Gampel Grinberg}}
%
%
%\begin{flushright}
%\emph{Fabiana Gampel Grinberg}
%\end{flushright}
%
%%\begin{multicols}{2}[]
%
%Este pequeno guia bilíngue
 %se propõe a acolher um ponto de vista amplo da espiritualidade que não se atenha somente ao pé da letra dos registros religiosos. Nossa proposta para este \emph{cabalat shabat} e  para a mensagem do texto de abertura dessa \emph{leket} ( {לקט}\,, ``coleção'', ``antologia'') diz respeito a aprender e compartilhar o que é significativo para cada um como indivíduo, dentre as muitas maneiras de celebração do judaísmo e de sua herança.%{Abençoado seja o Espírito do Universo, Fonte da Eternidade, que nos dá a habilidade de %questionar}.
%
%\medskip
%\emph{Todá rabá!}

%\end{multicols}