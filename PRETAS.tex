%!TEX root=./LIVRO.tex 
\textbf{Cabalat shabat: poemas rituais} apresenta o shabat como encenação simbólica da celebração do casamento entre Deus e o povo de Israel a partir da reunião de bênçãos e \emph{piyyutim} (poemas litúrgicos recitados em serviços religiosos) traduzidos nessa edição diretamente do hebraico. São cantos entoados durante o recebimento (em hebraico, \emph{cabalat}) do shabat, sinalizado pela primeira estrela que aparece no céu da sexta"-feira e que dura até o anoitecer do sábado. O termo \emph{piyyutim} vem do grego \emph{poiētḗs}, ponto de convergência que aproxima as duas antigas tradições poéticas através do texto introdutório do volume, e aponta principalmente para o gênero epitalâmio (em grego ``diante do tálamo'', ou seja, do quarto de dormir) como paralelo entre a estrutura grega e os textos rituais de shabat.

\textbf{Isaac Lúria}, o ``Ari'' (1534--1572), foi o maior e mais conhecido nome dentre os cabalistas. Nascido em Jerusalém, estabeleceu"-se na cidade de Tzfat (ou Safed) --- que se tornou o centro mundial da cabalá --- por volta de 1569. A palavra hebraica \emph{cabalat} vem de \emph{cabalá}, ``recebimento''. A cerimônia de \emph{cabalat shabat} como a conhecemos hoje tem grande influência da mística de Lúria e simboliza o recebimento da \emph{Schechiná}, literalmente, ``assentamento'', ``habitação'' ou ``moradia'', que em sentido alegórico significa ``presença de Deus''. A cabalá surgiu originalmente no sul da França, mas alcançou pleno desenvolvimento na Espanha do século \textsc{xiii}. Ganhou novo significado a partir do século \textsc{xvi} ao ser resignificada como uma espécie de resposta à questão do exílio dos judeus ibéricos em 1492. No século \textsc{xvii} foi vinculada ao movimento messiânico de Sabatai Tzvi, considerado herético à época, e que mesmo em colapso provocou a omissão da cabalá dos círculos judaicos oficiais.

\textbf{Fabiana Gampel Grinberg} é formada em Letras pela Universidade Federal de Minas Gerais e mestranda em Língua Hebraica, Literatura e Cultura Judaica, além de integrar, como pesquisadora, o Núcleo de Estudos em Judaísmo Contemporâneo da Universidade de São Paulo.



