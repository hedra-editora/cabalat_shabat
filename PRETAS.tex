%!TEX root=./LIVRO.tex 
\textbf{Cabalat shabat: poemas rituais} apresenta o \emph{shabat} como encenação simbólica da celebração do casamento entre Deus e o povo de Israel. A tradição da mística cabalista deu contornos bastante específicos ao ritual e transformou"-o em alegoria, utilizando"-se para isso de um lugar poético próximo ao gênero grego epitalâmio (``diante do tálamo'' ou do ``quarto de dormir''), que figura a partir dessa ótica um Deus mais próximo de quem o procura no âmbito doméstico, numa manobra alegórica que transforma um dos gêneros menos divinos em lugar de comunhão. Por fim, os cantos e bênçãos são dispostos de maneira secular mas não antirreligiosa, e trazem no próprio ritual um entendimento estrutural e histórico do judaísmo como simbologia.

\textbf{Isaac Lúria}, o ``Ari'' (1534--1572), foi o maior e mais conhecido nome dentre os cabalistas. Nascido em Jerusalém, estabeleceu"-se na cidade de Tzfat (ou Safed) --- que se tornou o centro mundial da cabala --- por volta de 1569. A palavra hebraica \emph{cabalat} vem de \emph{cabalá}, ``recebimento''. A cerimônia de \emph{cabalat shabat} como a conhecemos hoje tem grande influência da mística de Lúria e simboliza o recebimento da \emph{Schechiná}, literalmente, ``assentamento'', ``habitação'' ou ``moradia'', que em sentido alegórico significa ``presença de Deus''. A cabala surgiu originalmente no sul da França, mas alcançou pleno desenvolvimento na Espanha do século \versal{XIII}. Ganhou novo significado a partir do século \versal{XVI} ao ser resignificada como uma espécie de resposta à questão do exílio dos judeus ibéricos em 1492. No século \versal{XVII} foi vinculada ao movimento messiânico de Sabatai Tzvi, considerado herético à época, e que mesmo em colapso provocou a omissão da cabala dos círculos judaicos oficiais.

\textbf{Fabiana Gampel Grinberg} é formada em Letras pela Universidade Federal de Minas Gerais e mestranda em Língua Hebraica, Literatura e Cultura Judaica, além de integrar o Núcleo de Estudos em Judaísmo Contemporâneo da Universidade de São Paulo.



